% ======================================================================
% Chapter 5 — Discussion
% ======================================================================

\section{Discussion}
\label{sec:discussion}

This chapter discusses and interprets the experimental results reported in Chapter~4. In contrast to the results chapter, which focused strictly on reporting observable quantities, the present discussion aims to relate these observations to theoretical expectations, system design choices, and practical constraints of the implemented MIMO-audio demonstrator. The discussion is limited to the main operating mode of the system, namely spatial multiplexing with four transmit channels and eight receive channels (4T8R), 4-QAM modulation, zero-forcing-style channel estimation, MMSE equalization, and no channel coding.

\subsection{Scope and perspective of the discussion}
\label{sec:discussion_scope}

The primary objective of the implemented system is not to maximize communication performance in terms of spectral efficiency or error-rate optimality, but to provide a stable, repeatable, and transparent MIMO-OFDM demonstrator suitable for analysis and teaching. Consequently, the discussion emphasizes qualitative consistency with theory, internal coherence of the reported metrics, and the extent to which the observed results can be explained by known properties of MIMO-OFDM systems and acoustic transmission channels.

\subsection{Stability and repeatability of the demonstrator}
\label{sec:discussion_stability}

A central outcome of the presented results is the stable operation of the MIMO-audio demonstrator across repeated runs with identical parameter settings. The system reliably supports end-to-end transmission of text payloads while simultaneously exposing a rich set of intermediate physical-layer observables, including synchronization metrics, channel estimates, equalized symbols, rank-related quantities, and EVM measures.

The observed repeatability can be attributed to several design choices described in Chapter~3, such as deterministic framing, controlled random number generation during transmission, and centralized storage of receiver-side analysis data. From a teaching-oriented perspective, this stability is essential, as it enables meaningful comparisons between different parameter settings and transmission modes without being dominated by uncontrolled run-to-run variability.

\subsection{Interpretation of timing synchronization behavior}
\label{sec:discussion_sync}

The synchronization results presented in Chapter~4 show well-defined peaks in the Schmidl--Cox timing metric for all receive channels, albeit with channel-dependent variations in magnitude and sharpness. Such variations are expected in an acoustic multi-channel scenario, where individual microphone paths experience different gains, noise levels, and multipath conditions.

The presence of a clear dominant peak in the timing metric confirms that the chosen synchronization scheme remains effective in the audio-frequency domain and under multi-channel operation. The use of a single global frame start index, derived from the earliest reliable detection across channels, represents a pragmatic compromise between per-channel accuracy and system-wide consistency. This choice simplifies subsequent OFDM block alignment while maintaining sufficient robustness for the demonstrated operating conditions.

\subsection{Channel estimation characteristics: CIR and CTF}
\label{sec:discussion_channel}

The estimated channel impulse responses (CIRs) and channel transfer functions (CTFs) exhibit features that are consistent with indoor acoustic propagation. The CIR plots reveal energy spread over multiple taps, reflecting multipath components caused by reflections from walls, furniture, and other objects. Differences between transmit--receive channel pairs further highlight the spatial diversity inherent in the 4T8R configuration.

In the frequency domain, the CTF magnitude responses show pronounced frequency selectivity across subcarriers. Such behavior is typical for broadband acoustic channels and directly motivates the use of OFDM. The consistency between CIR and CTF representations confirms the internal coherence of the channel estimation pipeline and provides a meaningful basis for subsequent MIMO detection and performance evaluation.

\subsection{Spatial multiplexing and channel rank behavior}
\label{sec:discussion_rank}

The channel-rank analysis provides insight into the spatial degrees of freedom available for spatial multiplexing. In the considered 4T8R configuration, the estimated rank approaches the number of transmit channels for a large portion of subcarriers, indicating that the channel can, in principle, support four parallel spatial streams.

However, the rank is not uniformly maximal across all subcarriers. Subcarrier-dependent rank reductions, accompanied by small minimum singular values and increased condition numbers, indicate locally ill-conditioned channel realizations. These effects are expected in frequency-selective channels and highlight that spatial multiplexing performance can vary significantly across frequency, even within a single OFDM symbol.

The inclusion of singular-value-based conditioning measures complements the rank metric by providing a more nuanced view of channel quality. While rank alone indicates the number of usable spatial dimensions, the singular-value spread reflects how reliably these dimensions can be exploited in the presence of noise.

\subsection{Symbol-domain quality and EVM interpretation}
\label{sec:discussion_evm}

The constellation plots and EVM results offer a direct view of symbol-domain quality after MIMO equalization. The observed constellation spreading is consistent with the estimated channel conditions and the residual noise and interference present after equalization. Differences in EVM between spatial streams further reflect asymmetries in channel quality and effective post-equalization SNR.

The subcarrier-resolved EVM curves closely mirror the frequency selectivity observed in the CTF estimates, illustrating the strong link between channel frequency response and symbol-domain distortion. Excluding header symbols modulated with BPSK from the EVM computation ensures that the reported EVM values accurately reflect the quality of the QAM payload symbols, thereby improving the interpretability of the metric.

Overall, the EVM results provide a continuous and informative performance indicator that complements discrete measures such as BER, especially in scenarios where error-free decoding is occasionally achieved despite non-negligible symbol distortion.

\subsection{End-to-end payload recovery and uncoded BER}
\label{sec:discussion_payload}

The successful reconstruction of text payloads demonstrates that the implemented receiver chain operates coherently across all processing stages, from synchronization and channel estimation to equalization and demapping. The reported bit error counts and BER values, when reference data is available, reflect the combined effect of channel conditions, MIMO detection, and the absence of channel coding.

While uncoded transmission limits the achievable error performance, the results confirm that reliable payload recovery is feasible under the chosen operating conditions. This outcome aligns with the demonstrator’s primary goal of illustrating MIMO-OFDM concepts rather than achieving near-capacity performance.

\subsection{Limitations and inconsistencies of the current implementation}
\label{sec:discussion_limitations}

Several limitations and implementation inconsistencies are intentionally acknowledged. Although channel coding options exist at the transmitter side in the broader project, receiver-side channel decoding is not yet integrated in a unified manner. As a result, coding gain is not realized in the current evaluation, and BER results correspond to uncoded transmission.

Furthermore, certain auxiliary metrics, such as carrier-frequency-offset estimates or SNR-related quantities, are not consistently stored across all processing branches or operating modes. These inconsistencies reflect the evolutionary nature of the software development and do not undermine the validity of the reported results, but they highlight areas where further consolidation would improve completeness and clarity.

\subsection{Outlook and possible extensions}
\label{sec:discussion_outlook}

Several extensions naturally follow from the presented work. Integrating receiver-side channel decoding would enable systematic coded BER evaluations and facilitate comparisons between modulation and coding schemes. Additional operating modes, such as Alamouti or eigenmode transmission, could be analyzed more extensively using the existing visualization and analysis framework.

Finally, further automation of experiment logging and result export would support larger measurement campaigns and enhance the demonstrator’s value as a teaching and experimentation platform.