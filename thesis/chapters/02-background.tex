\section{Theoretical Background}

\subsection{Radio Propagation Channels and Motivation for OFDM}

In practical communication environments, signal propagation between transmitter and receiver is rarely limited to a single line-of-sight path. Instead, transmitted signals are reflected, diffracted, and scattered by surrounding objects, resulting in multiple propagation paths with different delays and attenuations. This phenomenon is commonly referred to as multipath propagation. Under the assumption that the channel remains constant over the observation interval, the propagation channel can be modeled as a linear time-invariant (LTI) system \cite{haring_ofdm_lecture}.

In this case, the channel is fully characterized by its channel impulse response (CIR), which can be expressed as
\begin{equation}
h_c(t) = \sum_{l=1}^{L} h_{c,l}\,\delta(t-\tau_l),
\end{equation}
where $h_{c,l}$ and $\tau_l$ denote the complex-valued path coefficient and propagation delay of the $l$-th multipath component, respectively, and $L$ represents the total number of propagation paths. The corresponding channel transfer function (CTF) in the frequency domain is obtained by the Fourier transform of the CIR and is given by
\begin{equation}
H_c(\omega) = \int_{-\infty}^{\infty} h_c(t)\,e^{-j\omega t}\,\mathrm{d}t
= \sum_{l=1}^{L} h_{c,l}\,e^{-j\omega \tau_l}.
\end{equation}
This representation highlights the frequency-selective nature of multipath channels and is commonly used in the analysis of multicarrier transmission systems \cite{proakis2008digital}.

A statistical characterization of the multipath channel is provided by the power delay profile (PDP), defined as the expected squared magnitude of the CIR:
\begin{equation}
P_{\mathrm{PDP}}(\tau) = \mathbb{E}\{|h_c(\tau)|^2\}.
\end{equation}
The PDP describes how the received signal energy is distributed over different delays and allows the definition of the maximum excess delay $\tau_{\max}$. Based on this parameter, the coherence bandwidth $B_c$ of the channel can be approximated as
\begin{equation}
B_c \approx \frac{1}{\tau_{\max}}.
\end{equation}
If the signal bandwidth $B$ is much smaller than $B_c$, the channel can be regarded as frequency-flat. Conversely, if $B$ exceeds $B_c$, the channel exhibits frequency-selective fading \cite{haring_ofdm_lecture}.

In the time domain, frequency-selective channels give rise to inter-symbol interference (ISI), since delayed replicas of previously transmitted symbols overlap with the current symbol. In single-carrier transmission systems, the mitigation of ISI typically requires sophisticated time-domain equalization techniques whose complexity increases significantly with the channel delay spread \cite{proakis2008digital}. This complexity motivates the use of alternative transmission schemes that can more efficiently cope with frequency-selective channels.

The fundamental idea of multicarrier transmission is to decompose a wideband frequency-selective channel into a set of narrowband subchannels, each of which experiences approximately flat fading. Orthogonal Frequency Division Multiplexing (OFDM) represents a practical and efficient realization of this concept. By transmitting data symbols in parallel over a large number of orthogonal subcarriers, OFDM significantly increases the symbol duration on each subcarrier and thereby reduces the impact of ISI \cite{nee2000ofdm}.

In OFDM systems, the subcarrier spacing $\Delta f$ is chosen as
\begin{equation}
\Delta f = \frac{1}{T_s},
\end{equation}
where $T_s$ denotes the useful OFDM symbol duration. This choice ensures orthogonality among the subcarriers under ideal synchronization conditions. To preserve subcarrier orthogonality in the presence of multipath propagation, a cyclic prefix (CP) is inserted at the beginning of each OFDM symbol. If the CP length is greater than or equal to the maximum channel delay spread, linear convolution with the channel impulse response is transformed into circular convolution. As a result, the frequency-domain channel matrix becomes diagonal, enabling simple one-tap equalization on each subcarrier \cite{haring_ofdm_lecture,nee2000ofdm}.

For discrete-time baseband modeling, the multipath channel is commonly represented as a finite impulse response (FIR) filter, also referred to as a tapped delay line model. This discrete-time representation provides the mathematical foundation for the implementation of OFDM systems using inverse discrete Fourier transform (IDFT) and discrete Fourier transform (DFT) operations, and it forms the basis for most practical OFDM transmitters and receivers.

\subsection{Principles of OFDM Transmission}

Orthogonal Frequency Division Multiplexing (OFDM) is a multicarrier transmission technique whose fundamental idea is to decompose a wideband frequency-selective channel into a set of mutually orthogonal narrowband subchannels. By doing so, each subchannel can be approximated as a frequency-flat channel, which significantly reduces the complexity of receiver-side equalization.

In practical systems, the implementation of OFDM relies on discrete-time signal processing, in particular on the discrete Fourier transform (DFT) and its fast algorithm (FFT). Compared to early multicarrier schemes based on analog oscillators, OFDM enables the realization of orthogonal subcarriers through digital signal processing, making it highly suitable for practical communication systems \cite{nee2000ofdm,haring_ofdm_lecture}.

\subsubsection{Discrete-Time OFDM Signal Model}

Consider a block of complex-valued data symbols
\[
\{X[k]\}_{k=0}^{N-1},
\]
to be transmitted over $N$ subcarriers within one OFDM symbol. These frequency-domain symbols are mapped onto orthogonal subcarriers and transformed into the time domain by means of an inverse discrete Fourier transform (IDFT). The resulting discrete-time OFDM signal can be expressed as
\begin{equation}
x[n] = \frac{1}{\sqrt{N}} \sum_{k=0}^{N-1} X[k]\,\mathrm{e}^{\mathrm{j}2\pi kn/N}, 
\quad n = 0,1,\dots,N-1.
\end{equation}

The normalization factor $1/\sqrt{N}$ ensures that the average signal power is preserved during the transformation. In practice, the IDFT operation is efficiently implemented using the inverse fast Fourier transform (IFFT).

The orthogonality of the subcarriers is guaranteed by an appropriate choice of the subcarrier spacing. Let $T_s$ denote the useful OFDM symbol duration. The subcarrier spacing is then defined as
\begin{equation}
\Delta f = \frac{1}{T_s}.
\end{equation}
Under ideal synchronization conditions, this choice ensures that the demodulation of one subcarrier does not introduce interference from other subcarriers.

\subsubsection{Cyclic Prefix and Circular Convolution}

When an OFDM signal is transmitted over a multipath channel, the transmitted signal undergoes linear convolution with the channel impulse response. Without further measures, this linear convolution destroys the orthogonality among subcarriers and leads to inter-symbol and inter-carrier interference.

To avoid this effect, a cyclic prefix (CP) is inserted at the beginning of each OFDM symbol. The CP is generated by copying the last $N_{\mathrm{CP}}$ samples of the OFDM symbol and appending them to its front.

If the CP length satisfies
\begin{equation}
N_{\mathrm{CP}} \geq L_h - 1,
\end{equation}
where $L_h$ denotes the length of the discrete-time channel impulse response, the linear convolution can be transformed into circular convolution after CP removal. In this case, the received discrete-time signal can be written as
\begin{equation}
y[n] = x[n] \circledast h[n] + w[n],
\end{equation}
where $\circledast$ denotes circular convolution and $w[n]$ represents additive noise \cite{proakis2008digital}.

\subsubsection{Frequency-Domain Representation and One-Tap Equalization}

Applying the discrete Fourier transform to the received OFDM symbol yields a frequency-domain signal model of the form
\begin{equation}
Y[k] = H[k]\,X[k] + W[k], 
\quad k = 0,1,\dots,N-1.
\end{equation}

This expression shows that, under ideal conditions, the equivalent channel in the frequency domain exhibits a diagonal structure. Each subcarrier is affected only by the corresponding frequency-domain channel coefficient $H[k]$, which allows independent processing of each subcarrier.

A simple equalization method is the zero-forcing (ZF) equalizer, given by
\begin{equation}
\hat{X}[k] = \frac{Y[k]}{H[k]},
\end{equation}
provided that $H[k] \neq 0$. In the presence of noise, more advanced techniques such as minimum mean square error (MMSE) equalization can be employed to achieve a better trade-off between noise enhancement and distortion \cite{proakis2008digital}.

\subsubsection{Advantages and Limitations of OFDM}

The main advantage of OFDM lies in its ability to efficiently combat frequency-selective fading with relatively low receiver complexity. By converting a frequency-selective channel into multiple parallel flat-fading subchannels, OFDM enables simple frequency-domain equalization and supports flexible allocation of spectral resources \cite{nee2000ofdm}.

Despite these advantages, OFDM also exhibits certain limitations. The superposition of many subcarriers results in a high peak-to-average power ratio (PAPR), which imposes stringent linearity requirements on the transmit power amplifier. Furthermore, OFDM systems are sensitive to synchronization errors, such as carrier frequency offset and timing misalignment. These impairments destroy subcarrier orthogonality and introduce inter-carrier interference (ICI), which must be carefully addressed in practical system implementations and experimental demonstrators \cite{haring_ofdm_lecture}.

\subsection{Fundamentals of MIMO Systems}

\subsubsection{Motivation for Multi-Antenna Communication}

In conventional single-input single-output (SISO) communication systems, system performance is primarily limited by channel fading and available bandwidth. In multipath propagation environments, random fluctuations in signal amplitude and phase may lead to deep fades and a significant degradation of link reliability.

Multi-antenna communication techniques address these limitations by introducing multiple antennas at the transmitter and/or receiver, thereby providing additional spatial degrees of freedom. Multiple-input multiple-output (MIMO) systems exploit these spatial dimensions to improve link reliability or increase data rates without requiring additional bandwidth or transmit power. As a result, MIMO has become a key technology in modern broadband communication systems.

\subsubsection{Classification of Multi-Antenna Systems}

Depending on the number of transmit and receive antennas, multi-antenna systems can be classified into four main categories: single-input single-output (SISO), single-input multiple-output (SIMO), multiple-input single-output (MISO), and multiple-input multiple-output (MIMO). While SIMO and MISO systems are primarily employed to achieve diversity or array gain, full MIMO systems are capable of supporting different transmission modes, including diversity-oriented and multiplexing-oriented schemes.

In practical system design, the choice of a specific MIMO transmission mode depends on the targeted performance objective, such as improved reliability or increased spectral efficiency.

\subsubsection{Transmission Modes and Gains in MIMO Systems}

Compared to single-antenna systems, MIMO systems offer several potential advantages. However, different MIMO transmission modes emphasize different performance objectives. In this work, \emph{spatial multiplexing} is considered the primary focus, while other transmission modes are introduced for comparison and contextual understanding.

\paragraph{Spatial Multiplexing (Focus of This Work)}

Spatial multiplexing is one of the most prominent transmission modes in MIMO systems. Its key principle is the simultaneous transmission of multiple independent data streams over the same time and frequency resources. Under favorable channel conditions, spatial multiplexing enables an almost linear increase in data rate with the number of antennas, making it particularly attractive for improving spectral efficiency.

In spatial multiplexing systems, independent data streams are transmitted from different antennas and superimposed in the spatial channel. At the receiver, multi-antenna signal processing algorithms are employed to separate these streams. Typical detection techniques include zero-forcing (ZF), minimum mean square error (MMSE) detection, and successive interference cancellation schemes such as the Vertical Bell Labs Layered Space-Time (V-BLAST) architecture.

Since the acoustic MIMO demonstrator considered in this thesis aims at algorithmic analysis and visualization, spatial multiplexing allows for an intuitive illustration of channel rank, spatial degrees of freedom, and receiver detection performance. Therefore, it is selected as the primary transmission mode implemented and analyzed in this work.

\paragraph{Diversity Transmission and Alamouti Coding}

In addition to spatial multiplexing, MIMO systems can also be operated in diversity-oriented modes to enhance link reliability. A prominent example is the Alamouti space-time block code, which achieves full diversity gain for two transmit antennas while maintaining a simple linear receiver structure.

Unlike spatial multiplexing, Alamouti coding does not aim to increase the data rate but rather improves robustness against fading through redundant transmission. In this work, Alamouti transmission is mainly considered as a reference scheme to highlight the trade-off between reliability and throughput in different MIMO transmission modes.

\paragraph{Eigenmode Transmission and Beamforming}

If channel state information is available at the transmitter, the MIMO channel matrix can be decomposed into independent spatial subchannels by means of eigenvalue or singular value decomposition. This approach, commonly referred to as eigenmode transmission or beamforming, enables capacity-optimal transmission under ideal conditions.

However, eigenmode transmission requires accurate channel knowledge at the transmitter and involves increased implementation complexity. Therefore, this transmission mode is not considered for practical implementation in the demonstrator, but is briefly introduced as part of the theoretical background.

\subsubsection{MIMO Channel Model and Matrix Representation}

For analytical purposes, MIMO systems are commonly described using a matrix-based baseband equivalent channel model. Assuming $N_\mathrm{T}$ transmit antennas and $N_\mathrm{R}$ receive antennas, the MIMO system can be expressed as
\begin{equation}
\mathbf{y} = \mathbf{H}\mathbf{x} + \mathbf{n},
\end{equation}
where $\mathbf{x} \in \mathbb{C}^{N_\mathrm{T} \times 1}$ denotes the transmit signal vector, $\mathbf{y} \in \mathbb{C}^{N_\mathrm{R} \times 1}$ the receive signal vector, $\mathbf{H} \in \mathbb{C}^{N_\mathrm{R} \times N_\mathrm{T}}$ the channel matrix, and $\mathbf{n}$ additive noise.

This model captures the fundamental property of MIMO systems, namely that multiple transmitted signals are linearly superimposed by the spatial channel and jointly processed at the receiver.

\subsubsection{Channel Rank and Spatial Degrees of Freedom}

An essential parameter in MIMO systems is the rank of the channel matrix. The channel rank determines the number of independent spatial data streams that can be transmitted simultaneously and is bounded by
\begin{equation}
\mathrm{rank}(\mathbf{H}) \leq \min(N_\mathrm{T}, N_\mathrm{R}).
\end{equation}

In spatial multiplexing systems, the channel rank directly limits the achievable multiplexing gain. A full-rank channel allows the transmission of the maximum number of independent data streams, whereas a reduced rank leads to a loss of spatial degrees of freedom. Consequently, the channel rank plays a central role in the analysis of spatial multiplexing performance and serves as a key metric in the evaluation of MIMO demonstrators.

\subsubsection{Summary and Relation to Subsequent Chapters}

This section introduced the fundamental concepts of multi-antenna communication with a focus on spatial multiplexing as the primary transmission mode. Diversity-oriented schemes such as Alamouti coding and eigenmode transmission were discussed for comparison. The matrix-based MIMO channel model and the concept of channel rank were presented as essential tools for analyzing spatial multiplexing systems.

In the following sections, these concepts are combined with OFDM transmission, leading to a per-subcarrier MIMO signal model and forming the basis for channel estimation, receiver-side detection algorithms, and performance evaluation in MIMO-OFDM systems.

\subsection{MIMO-OFDM Signal Model}

\subsubsection{Conceptual Combination of MIMO and OFDM}

As introduced in the previous sections, OFDM enables the decomposition of a frequency-selective wideband channel into a set of parallel frequency-flat subchannels, while MIMO systems exploit spatial degrees of freedom by employing multiple transmit and receive antennas. The combination of these two techniques results in a MIMO-OFDM system, which has become the standard transmission scheme for modern broadband communication systems \cite{nee2000ofdm,tse2005fundamentals}.

The key idea of MIMO-OFDM is to apply a narrowband MIMO channel model independently to each OFDM subcarrier. Since OFDM converts a frequency-selective channel into multiple approximately flat-fading subchannels, the complex wideband MIMO channel can be represented as a collection of parallel narrowband MIMO channels in the frequency domain \cite{proakis2008digital,haring_ofdm_lecture}. This per-subcarrier modeling approach significantly simplifies the analysis and receiver design of broadband MIMO systems.

\subsubsection{Discrete-Time Transmit Model}

Consider a MIMO-OFDM system with $N_\mathrm{T}$ transmit antennas and $N_\mathrm{R}$ receive antennas employing $N$ subcarriers per OFDM symbol. In spatial multiplexing mode, $N_\mathrm{S}$ independent data streams are transmitted simultaneously, where
\[
N_\mathrm{S} \leq \min(N_\mathrm{T}, N_\mathrm{R}).
\]

On the $k$-th subcarrier, the transmit signal can be represented by the vector
\[
\mathbf{x}[k] \in \mathbb{C}^{N_\mathrm{T} \times 1},
\]
whose elements correspond to the complex-valued symbols transmitted from the individual antennas on that subcarrier. After OFDM modulation, including IFFT operation and cyclic prefix insertion, the time-domain signals of all transmit antennas are emitted simultaneously and propagated through the spatial channel.

\subsubsection{Per-Subcarrier MIMO Channel Model}

At the receiver, after cyclic prefix removal and FFT processing, the received signal on the $k$-th subcarrier can be expressed as the vector
\[
\mathbf{y}[k] \in \mathbb{C}^{N_\mathrm{R} \times 1}.
\]

Under ideal synchronization conditions, the MIMO-OFDM system on each subcarrier is described by the linear model
\begin{equation}
\mathbf{y}[k] = \mathbf{H}[k]\mathbf{x}[k] + \mathbf{n}[k],
\label{eq:mimo_ofdm_model}
\end{equation}
where $\mathbf{H}[k] \in \mathbb{C}^{N_\mathrm{R} \times N_\mathrm{T}}$ denotes the frequency-domain MIMO channel matrix on the $k$-th subcarrier and $\mathbf{n}[k]$ represents additive noise.

This expression constitutes the standard per-subcarrier signal model used for the analysis of MIMO-OFDM systems and is widely adopted in the literature and textbooks on multicarrier and multi-antenna communications \cite{proakis2008digital,tse2005fundamentals,haring_ofdm_lecture}. It highlights that OFDM transforms the convolutional wideband MIMO channel into a set of independent linear MIMO systems in the frequency domain.

\subsubsection{Structure and Physical Interpretation of the Channel Matrix}

Each element $H_{ij}[k]$ of the channel matrix $\mathbf{H}[k]$ represents the complex frequency response between the $j$-th transmit antenna and the $i$-th receive antenna on the $k$-th subcarrier. The channel matrix therefore captures the combined effects of multipath propagation, antenna configuration, and the spatial characteristics of the propagation environment.

In spatial multiplexing systems, the rank of $\mathbf{H}[k]$ determines the number of independent spatial data streams that can be transmitted on the $k$-th subcarrier. A full-rank channel matrix allows the system to fully exploit spatial degrees of freedom, whereas a rank-deficient channel limits the achievable spatial multiplexing gain. As a consequence, variations of the channel rank across subcarriers directly influence the frequency-dependent performance of MIMO-OFDM systems \cite{tse2005fundamentals}.

\subsubsection{Detection Problem in MIMO-OFDM Systems}

In spatial multiplexing mode, the primary task of the receiver is to recover the transmit vector $\mathbf{x}[k]$ from the received signal $\mathbf{y}[k]$. This task is commonly referred to as MIMO detection.

Due to the coupling of multiple data streams through the channel matrix $\mathbf{H}[k]$, MIMO detection constitutes a multidimensional signal separation problem. Depending on the employed algorithm, different trade-offs between computational complexity and detection performance arise. Linear detection schemes, such as zero-forcing (ZF) and minimum mean square error (MMSE) detection, offer low complexity, while successive interference cancellation techniques, including the Vertical Bell Labs Layered Space-Time (V-BLAST) algorithm, generally achieve improved performance at the cost of increased complexity \cite{tse2005fundamentals}.

In MIMO-OFDM systems, the detection process is typically performed independently on each subcarrier, which allows the wideband detection problem to be decomposed into a set of parallel narrowband MIMO detection problems.

\subsubsection{Summary and Relation to Subsequent Chapters}

This section established a unified signal model for MIMO-OFDM systems based on a per-subcarrier representation. The structure and physical interpretation of the frequency-domain channel matrix were discussed, and the role of channel rank in spatial multiplexing systems was highlighted. The resulting signal model provides the theoretical foundation for channel estimation, equalization, and MIMO detection algorithms, which are addressed in the subsequent sections.

\subsection{Channel Estimation and Equalization in MIMO-OFDM Systems}

\subsubsection{Role of Channel Estimation in MIMO-OFDM}

As established in the previous section, the received signal on the $k$-th subcarrier of a MIMO-OFDM system can be expressed as
\begin{equation}
\mathbf{y}[k] = \mathbf{H}[k]\mathbf{x}[k] + \mathbf{n}[k].
\end{equation}
In order to reliably recover the transmit vector $\mathbf{x}[k]$, knowledge of the corresponding channel matrix $\mathbf{H}[k]$ is required at the receiver. Consequently, channel estimation constitutes a key component of MIMO-OFDM receivers, and its accuracy has a direct impact on equalization and detection performance \cite{proakis2008digital,tse2005fundamentals}.

Due to the OFDM structure, channel estimation can be performed independently on each subcarrier in the frequency domain, which significantly reduces computational complexity compared to time-domain approaches for wideband systems.

\subsubsection{Training-Based Channel Estimation}

One of the most fundamental channel estimation approaches is based on known training sequences or preambles transmitted prior to data transmission. By exploiting the knowledge of the transmitted symbols, the receiver can directly estimate the channel response from the received signal.

For a MIMO-OFDM system, assuming that a known training symbol matrix $\mathbf{X}[k]$ is transmitted on the $k$-th subcarrier, the received signal can be written as
\begin{equation}
\mathbf{Y}[k] = \mathbf{H}[k]\mathbf{X}[k] + \mathbf{N}[k].
\end{equation}
If $\mathbf{X}[k]$ is full rank and known at the receiver, a least-squares (LS) estimate of the channel matrix is given by
\begin{equation}
\hat{\mathbf{H}}[k] = \mathbf{Y}[k]\mathbf{X}^{-1}[k].
\end{equation}

This approach is simple to implement and provides reliable performance at sufficiently high signal-to-noise ratios. Therefore, training-based channel estimation is well suited for experimental platforms and teaching-oriented demonstrators \cite{haring_ofdm_lecture}.

\subsubsection{Pilot-Based Channel Estimation}

In practical systems with time-varying channels, relying solely on preamble-based estimation is often insufficient. To enable continuous channel tracking, pilot symbols are periodically embedded into the transmitted data stream.

In OFDM systems, pilots can be arranged in the time domain, frequency domain, or both. The channel response is first estimated at the pilot positions and subsequently interpolated across data subcarriers. In MIMO-OFDM systems, pilot design must additionally ensure orthogonality among transmit antennas to avoid pilot contamination and to maintain channel identifiability \cite{nee2000ofdm}.

\subsubsection{Frequency-Domain Equalization}

Once an estimate $\hat{\mathbf{H}}[k]$ of the channel matrix is available, the receiver compensates for channel-induced distortions by means of equalization. Owing to the cyclic prefix, the equivalent frequency-domain channel exhibits a per-subcarrier matrix structure, which allows equalization to be performed independently on each subcarrier.

In spatial multiplexing mode, equalization aims to construct a matrix $\mathbf{W}[k]$ such that the estimate
\begin{equation}
\hat{\mathbf{x}}[k] = \mathbf{W}[k]\mathbf{y}[k]
\end{equation}
approximates the original transmit vector $\mathbf{x}[k]$ as accurately as possible.

\subsubsection{Linear Equalization: ZF and MMSE}

The most commonly employed equalization techniques are linear equalizers, in particular zero-forcing (ZF) and minimum mean square error (MMSE) equalization.

The ZF equalizer completely eliminates inter-stream interference by inverting the channel matrix, yielding
\begin{equation}
\mathbf{W}_{\mathrm{ZF}}[k] = \hat{\mathbf{H}}^{-1}[k],
\end{equation}
provided that the channel matrix is invertible. While ZF equalization achieves perfect interference suppression in the absence of noise, it may significantly amplify noise under unfavorable channel conditions.

In contrast, MMSE equalization accounts for noise by balancing interference suppression and noise enhancement. The MMSE equalization matrix is given by
\begin{equation}
\mathbf{W}_{\mathrm{MMSE}}[k]
= \left( \hat{\mathbf{H}}^{\mathrm{H}}[k]\hat{\mathbf{H}}[k] + \sigma_n^2 \mathbf{I} \right)^{-1}
\hat{\mathbf{H}}^{\mathrm{H}}[k],
\end{equation}
where $\sigma_n^2$ denotes the noise variance. MMSE equalization generally outperforms ZF at low and moderate signal-to-noise ratios \cite{tse2005fundamentals}.

\subsubsection{Successive Interference Cancellation and V-BLAST}

To further improve detection performance in spatial multiplexing systems, successive interference cancellation (SIC) techniques can be applied. A prominent example is the Vertical Bell Labs Layered Space-Time (V-BLAST) architecture.

The core idea of V-BLAST is to detect the data stream with the highest post-detection signal-to-noise ratio first, subtract its contribution from the received signal, and subsequently detect the remaining streams. By iteratively reducing inter-stream interference, V-BLAST achieves improved performance compared to purely linear detection methods, while maintaining manageable computational complexity \cite{tse2005fundamentals}.

In MIMO-OFDM systems, V-BLAST detection is typically performed independently on each subcarrier, preserving the modular structure of OFDM-based receivers.

\subsubsection{Performance Metrics for Receiver Evaluation}

Following channel estimation, equalization, and detection, quantitative performance metrics are required to evaluate receiver quality. One of the most fundamental measures is the signal-to-noise ratio (SNR), defined as the ratio between signal power and noise power:
\begin{equation}
\mathrm{SNR} = \frac{P_{\mathrm{signal}}}{P_{\mathrm{noise}}}.
\end{equation}
In MIMO-OFDM systems, SNR can be defined at different stages of the receiver chain and provides an important indication of channel conditions and operating points.

Another widely used metric is the error vector magnitude (EVM), which quantifies the deviation between received symbols and their ideal reference positions. For the $k$-th subcarrier, the error vector is defined as
\begin{equation}
\mathbf{e}[k] = \hat{\mathbf{x}}[k] - \mathbf{x}[k].
\end{equation}
The root-mean-square EVM is then given by
\begin{equation}
\mathrm{EVM} =
\sqrt{
\frac{\mathbb{E}\{\|\mathbf{e}[k]\|^2\}}
     {\mathbb{E}\{\|\mathbf{x}[k]\|^2\}}
}.
\end{equation}

Unlike bit error rate (BER), EVM directly reflects symbol-level distortions and is particularly sensitive to residual interference, noise, and synchronization errors. Therefore, EVM is well suited for analyzing modulation quality and receiver performance in experimental and demonstrator-based systems \cite{proakis2008digital}.

In addition, constellation diagrams provide an intuitive visualization of received symbol distributions in the complex plane and complement quantitative performance metrics. For spatial multiplexing systems, the rank of the channel matrix further serves as an important indicator of available spatial degrees of freedom and achievable multiplexing performance \cite{tse2005fundamentals}.

\subsubsection{Summary and Relation to System Evaluation}

This section discussed channel estimation and equalization techniques for MIMO-OFDM systems, including training-based and pilot-based estimation as well as ZF, MMSE, and V-BLAST detection methods. Furthermore, relevant performance metrics such as SNR, EVM, constellation diagrams, and channel rank were introduced. These concepts form the theoretical basis for the performance evaluation and experimental analysis presented in subsequent chapters.
