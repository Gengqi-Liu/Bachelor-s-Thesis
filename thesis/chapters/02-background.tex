\section{Theoretical Background}

\subsection{Radio Propagation Channels and Motivation for OFDM}

In practical communication environments, signal propagation between transmitter and receiver is rarely limited to a single line-of-sight path. Instead, transmitted signals are reflected, diffracted, and scattered by surrounding objects, resulting in multiple propagation paths with different delays and attenuations. This phenomenon is commonly referred to as multipath propagation. Under the assumption that the channel remains constant over the observation interval, the propagation channel can be modeled as a linear time-invariant (LTI) system \cite{haring_ofdm_lecture}.

In this case, the channel is fully characterized by its channel impulse response (CIR), which can be expressed as
\begin{equation}
h_{\mathrm{c}}(t) = \sum_{l=1}^{L} h_{{\mathrm{c}},l}\,\delta(t-\tau_l),
\end{equation}
where $h_{{\mathrm{c}},l}$ and $\tau_l$ denote the complex-valued path coefficient and propagation delay of the $l$-th multipath component, respectively, and $L$ represents the total number of propagation paths. The corresponding channel transfer function (CTF) in the frequency domain is obtained by the Fourier transform of the CIR and is given by
\begin{equation}
H_{\mathrm{c}}(\omega) = \int_{-\infty}^{\infty} h_{\mathrm{c}}(t)\,\mathrm{e}^{-\mathrm{j}\omega t}\,\mathrm{d}t
= \sum_{l=1}^{L} h_{{\mathrm{c}},l}\,\mathrm{e}^{-\mathrm{j}\omega \tau_l}.
\end{equation}
This representation highlights the frequency-selective nature of multipath channels and is commonly used in the analysis of multicarrier transmission systems \cite{proakis2008digital}.

A statistical characterization of the multipath channel is provided by the power delay profile (PDP), defined as the expected squared magnitude of the CIR:
\begin{equation}
P_{\mathrm{PDP}}(\tau) = \mathbb{E}\{|h_{\mathrm{c}}(\tau)|^2\}.
\end{equation}
The PDP describes how the received signal energy is distributed over different delays and allows the definition of the maximum excess delay $\tau_{\max}$. Based on this parameter, the coherence bandwidth $B_c$ of the channel can be approximated as
\begin{equation}
B_{\mathrm{c}} \approx \frac{1}{\tau_{\max}}.
\end{equation}
If the signal bandwidth $B$ is much smaller than $B_{\mathrm{c}}$, the channel can be regarded as frequency-flat. Conversely, if $B$ exceeds $B_{\mathrm{c}}$, the channel exhibits frequency-selective fading \cite{haring_ofdm_lecture}.

In the time domain, frequency-selective channels give rise to inter-symbol interference (ISI), since delayed replicas of previously transmitted symbols overlap with the current symbol. In single-carrier transmission systems, the mitigation of ISI typically requires sophisticated time-domain equalization techniques whose complexity increases significantly with the channel delay spread \cite{proakis2008digital}. This complexity motivates the use of alternative transmission schemes that can more efficiently cope with frequency-selective channels.

The fundamental idea of multicarrier transmission is to decompose a wideband frequency-selective channel into a set of narrowband subchannels, each of which experiences approximately flat fading. Orthogonal Frequency Division Multiplexing (OFDM) represents a practical and efficient realization of this concept. By transmitting data symbols in parallel over a large number of orthogonal subcarriers, OFDM significantly increases the symbol duration on each subcarrier and thereby reduces the impact of ISI \cite{nee2000ofdm}.

In OFDM systems, the subcarrier spacing $\Delta f$ is chosen as
\begin{equation}
\Delta f = \frac{1}{T_s},
\end{equation}
where $T_s$ denotes the useful OFDM symbol duration. This choice ensures orthogonality among the subcarriers under ideal synchronization conditions. To preserve subcarrier orthogonality in the presence of multipath propagation, a cyclic prefix (CP) is inserted at the beginning of each OFDM symbol. If the CP length is greater than or equal to the maximum channel delay spread, linear convolution with the channel impulse response is transformed into circular convolution. As a result, the frequency-domain channel matrix becomes diagonal, enabling simple one-tap equalization on each subcarrier \cite{haring_ofdm_lecture,nee2000ofdm}.

For discrete-time baseband modeling, the multipath channel is commonly represented as a finite impulse response (FIR) filter, also referred to as a tapped delay line model. This discrete-time representation provides the mathematical foundation for the implementation of OFDM systems using inverse discrete Fourier transform (IDFT) and discrete Fourier transform (DFT) operations, and it forms the basis for most practical OFDM transmitters and receivers.

\begin{figure}[t]
    \centering
    \includegraphics[width=0.95\linewidth]{figures/ch2/ofdm_tx_rx_blockdiagram.png}
    \caption{Block diagram of a generic OFDM transmitter and receiver chain, illustrating the main signal-processing stages and the role of time and frequency synchronization. Adapted from \cite{haring_ofdm_lecture}.}
    \label{fig:ofdm_blockdiagram}
\end{figure}

Figure~\ref{fig:ofdm_blockdiagram} illustrates a generic block diagram of an OFDM transmitter and receiver, highlighting the key signal-processing stages discussed above. In particular, the diagram emphasizes how time and frequency synchronization, cyclic prefix handling, and frequency-domain equalization interact to mitigate the effects of multipath propagation and enable efficient multicarrier transmission.

\subsection{Principles of OFDM Transmission}

Orthogonal Frequency Division Multiplexing (OFDM) is a multicarrier transmission technique whose fundamental idea is to decompose a wideband frequency-selective channel into a set of mutually orthogonal narrowband subchannels. By doing so, each subchannel can be approximated as a frequency-flat channel, which significantly reduces the complexity of receiver-side equalization.

In practical systems, the implementation of OFDM relies on discrete-time signal processing, in particular on the discrete Fourier transform (DFT) and its fast algorithm (FFT). Compared to early multicarrier schemes based on analog oscillators, OFDM enables the realization of orthogonal subcarriers through digital signal processing, making it highly suitable for practical communication systems \cite{nee2000ofdm,haring_ofdm_lecture}.

\subsubsection{Discrete-Time OFDM Signal Model}

Consider a block of complex-valued data symbols
\[
\{X[k]\}_{k=0}^{N-1},
\]
to be transmitted over $N$ subcarriers within one OFDM symbol. These frequency-domain symbols are mapped onto orthogonal subcarriers and transformed into the time domain by means of an inverse discrete Fourier transform (IDFT). The resulting discrete-time OFDM signal can be expressed as
\begin{equation}
x[n] = \frac{1}{\sqrt{N}} \sum_{k=0}^{N-1} X[k]\,\mathrm{e}^{\mathrm{j}2\pi kn/N}, 
\quad n = 0,1,\dots,N-1.
\end{equation}

The normalization factor $1/\sqrt{N}$ ensures that the average signal power is preserved during the transformation. In practice, the IDFT operation is efficiently implemented using the inverse fast Fourier transform (IFFT).

The orthogonality of the subcarriers is guaranteed by an appropriate choice of the subcarrier spacing. Let $T_s$ denote the useful OFDM symbol duration. The subcarrier spacing is then defined as
\begin{equation}
\Delta f = \frac{1}{T_s}.
\end{equation}
Under ideal synchronization conditions, this choice ensures that the demodulation of one subcarrier does not introduce interference from other subcarriers.

\subsubsection{Cyclic Prefix and Circular Convolution}

When an OFDM signal is transmitted over a multipath channel, the transmitted signal undergoes linear convolution with the channel impulse response. Without further measures, this linear convolution destroys the orthogonality among subcarriers and leads to inter-symbol and inter-carrier interference.

To avoid this effect, a cyclic prefix (CP) is inserted at the beginning of each OFDM symbol. The CP is generated by copying the last $N_{\mathrm{CP}}$ samples of the OFDM symbol and appending them to its front.

If the CP length satisfies
\begin{equation}
N_{\mathrm{CP}} \geq L_{\mathrm{h}} - 1,
\end{equation}
where $L_{\mathrm{h}}$ denotes the length of the discrete-time channel impulse response, the linear convolution can be transformed into circular convolution after CP removal. In this case, the received discrete-time signal can be written as
\begin{equation}
y[n] = x[n] \circledast h[n] + w[n],
\end{equation}
where $\circledast$ denotes circular convolution and $w[n]$ represents additive noise \cite{proakis2008digital}.

\subsubsection{Frequency-Domain Representation and One-Tap Equalization}

Applying the discrete Fourier transform to the received OFDM symbol yields a frequency-domain signal model of the form
\begin{equation}
Y[k] = H[k]\,X[k] + W[k], 
\quad k = 0,1,\dots,N-1.
\end{equation}

This expression shows that, under ideal conditions, the equivalent channel in the frequency domain exhibits a diagonal structure. Each subcarrier is affected only by the corresponding frequency-domain channel coefficient $H[k]$, which allows independent processing of each subcarrier.

A simple equalization method is the zero-forcing (ZF) equalizer, given by
\begin{equation}
\hat{X}[k] = \frac{Y[k]}{H[k]},
\end{equation}
provided that $H[k] \neq 0$. In the presence of noise, more advanced techniques such as minimum mean square error (MMSE) equalization can be employed to achieve a better trade-off between noise enhancement and distortion \cite{proakis2008digital}.

\subsubsection{Advantages and Limitations of OFDM}

The main advantage of OFDM lies in its ability to efficiently combat frequency-selective fading with relatively low receiver complexity. By converting a frequency-selective channel into multiple parallel flat-fading subchannels, OFDM enables simple frequency-domain equalization and supports flexible allocation of spectral resources \cite{nee2000ofdm}.

Despite these advantages, OFDM also exhibits certain limitations. The superposition of many subcarriers results in a high peak-to-average power ratio (PAPR), which imposes stringent linearity requirements on the transmit power amplifier. Furthermore, OFDM systems are sensitive to synchronization errors, such as carrier frequency offset and timing misalignment. These impairments destroy subcarrier orthogonality and introduce inter-carrier interference (ICI), which must be carefully addressed in practical system implementations and experimental demonstrators \cite{haring_ofdm_lecture}.

\subsection{Fundamentals of MIMO Systems}

\subsubsection{Motivation for Multi-Antenna Communication}

In conventional single-input single-output (SISO) communication systems, performance is fundamentally limited by channel fading and available bandwidth. In multipath propagation environments, random fluctuations in signal amplitude and phase may lead to deep fades, resulting in a significant degradation of link reliability.

Multi-antenna communication techniques address these limitations by introducing multiple antennas at the transmitter and/or receiver, thereby creating additional spatial degrees of freedom. Multiple-input multiple-output (MIMO) systems exploit these spatial dimensions to either improve link reliability through diversity or increase data rates through parallel transmission, without requiring additional bandwidth or transmit power. For this reason, MIMO has become a key enabling technology in modern broadband communication systems.

From a system-level perspective, the use of multiple antennas naturally generalizes single-antenna configurations. Depending on the number of transmit and receive antennas, systems may operate in SISO, SIMO, MISO, or full MIMO configurations. In the context of the audio-based demonstrator developed in this work, all of these configurations can be realized, with the focus placed on full MIMO operation.

\subsubsection{Transmission Modes in MIMO Systems}

While the presence of multiple antennas provides additional spatial degrees of freedom, the manner in which these degrees of freedom are exploited depends on the chosen transmission mode. Different MIMO transmission strategies emphasize different performance objectives, such as reliability, throughput, or robustness against channel impairments.

\paragraph{Spatial Multiplexing (Primary Focus of This Work)}

Spatial multiplexing aims at increasing the achievable data rate by transmitting multiple independent data streams simultaneously over the same time and frequency resources. Under favorable channel conditions, the achievable throughput can scale approximately linearly with the number of transmit antennas, making spatial multiplexing particularly attractive for spectrally efficient communication.

In spatial multiplexing systems, independent data streams are transmitted from different antennas and superimposed by the propagation channel. At the receiver, multi-antenna signal processing algorithms are required to separate the streams. Common detection techniques include zero-forcing (ZF), minimum mean square error (MMSE) detection, and successive interference cancellation schemes such as the V-BLAST architecture.

For the audio-based MIMO demonstrator considered in this thesis, spatial multiplexing is especially well suited for algorithmic analysis and visualization. It enables a direct illustration of channel rank, spatial degrees of freedom, and receiver detection performance, and is therefore selected as the primary transmission mode for implementation and evaluation.

\paragraph{Diversity-Oriented Transmission and Alamouti Coding}

In contrast to spatial multiplexing, diversity-oriented transmission schemes aim at improving link robustness rather than increasing data rate. A well-known example is the Alamouti space-time block code, which achieves full transmit diversity for two transmit antennas while retaining a simple linear receiver structure.

In this work, Alamouti transmission is considered mainly as a reference scheme. It provides a useful contrast to spatial multiplexing by highlighting the fundamental trade-off between reliability and throughput in different MIMO transmission modes.

\paragraph{Eigenmode Transmission and Beamforming}

If accurate channel state information is available at the transmitter, the MIMO channel matrix can be decomposed into orthogonal spatial subchannels using singular value decomposition. This principle, commonly referred to as eigenmode transmission or beamforming, enables capacity-optimal transmission under idealized conditions.

Due to its reliance on transmitter-side channel knowledge and increased implementation complexity, eigenmode transmission is not pursued as a primary operating mode in the demonstrator. Nevertheless, it is briefly introduced here to complete the conceptual overview of MIMO transmission strategies.

\subsubsection{MIMO Channel Model and Spatial Degrees of Freedom}

For analytical purposes, MIMO systems are commonly described using a matrix-based baseband-equivalent channel model. Assuming $N_\mathrm{T}$ transmit antennas and $N_\mathrm{R}$ receive antennas, the input--output relationship can be written as
\begin{equation}
\mathbf{y} = \mathbf{H}\mathbf{x} + \mathbf{n},
\end{equation}
where $\mathbf{x} \in \mathbb{C}^{N_\mathrm{T} \times 1}$ denotes the transmit signal vector, $\mathbf{y} \in \mathbb{C}^{N_\mathrm{R} \times 1}$ the receive signal vector, $\mathbf{H} \in \mathbb{C}^{N_\mathrm{R} \times N_\mathrm{T}}$ the channel matrix, and $\mathbf{n}$ additive noise.

An essential property of the MIMO channel is its rank, which determines the number of independent spatial data streams that can be transmitted simultaneously. The channel rank is bounded by
\begin{equation}
\mathrm{rank}(\mathbf{H}) \leq \min(N_\mathrm{T}, N_\mathrm{R}).
\end{equation}

In spatial multiplexing systems, the channel rank directly limits the achievable multiplexing gain. Consequently, rank-related metrics play a central role in the analysis and evaluation of MIMO-OFDM systems and form a key part of the experimental results presented in later chapters.

\subsection{MIMO-OFDM Signal Model}

\subsubsection{Conceptual Combination of MIMO and OFDM}

OFDM enables the decomposition of a frequency-selective wideband channel into a set of parallel frequency-flat subchannels, while MIMO systems exploit spatial degrees of freedom by employing multiple transmit and receive antennas. The combination of these two techniques results in a MIMO-OFDM system, which has become the dominant transmission scheme in modern broadband communication systems \cite{nee2000ofdm,tse2005fundamentals}.

The key idea of MIMO-OFDM is to apply a narrowband MIMO channel model independently to each OFDM subcarrier. By transforming a frequency-selective channel into multiple approximately flat-fading subchannels, OFDM allows the wideband MIMO channel to be represented as a collection of parallel narrowband MIMO channels in the frequency domain. This per-subcarrier representation significantly simplifies both analytical treatment and receiver design \cite{proakis2008digital,haring_ofdm_lecture}.

\subsubsection{Discrete-Time Transmit Model}

Consider a MIMO-OFDM system with $N_\mathrm{T}$ transmit antennas and $N_\mathrm{R}$ receive antennas employing $N$ subcarriers per OFDM symbol. In spatial multiplexing mode, $N_\mathrm{S}$ independent data streams are transmitted simultaneously, where
\[
N_\mathrm{S} \leq \min(N_\mathrm{T}, N_\mathrm{R}).
\]

On the $k$-th subcarrier, the transmitted signal is represented by the vector
\[
\mathbf{x}[k] \in \mathbb{C}^{N_\mathrm{T} \times 1},
\]
whose elements correspond to the complex-valued symbols transmitted from the individual antennas. After OFDM modulation, including IFFT processing and cyclic prefix insertion, the time-domain signals of all transmit antennas are emitted simultaneously and propagated through the spatial channel.

\subsubsection{Per-Subcarrier MIMO Channel Model}

At the receiver, following cyclic prefix removal and FFT processing, the received signal on the $k$-th subcarrier can be expressed as
\[
\mathbf{y}[k] \in \mathbb{C}^{N_\mathrm{R} \times 1}.
\]

Under ideal synchronization conditions, the input--output relationship on each subcarrier is described by the linear model
\begin{equation}
\mathbf{y}[k] = \mathbf{H}[k]\mathbf{x}[k] + \mathbf{n}[k],
\label{eq:mimo_ofdm_model}
\end{equation}
where $\mathbf{H}[k] \in \mathbb{C}^{N_\mathrm{R} \times N_\mathrm{T}}$ denotes the frequency-domain MIMO channel matrix and $\mathbf{n}[k]$ additive noise.

This formulation shows that OFDM converts the convolutional wideband MIMO channel into a set of independent linear MIMO systems in the frequency domain, each of which can be processed separately \cite{proakis2008digital,tse2005fundamentals}.

\subsubsection{Structure and Interpretation of the Channel Matrix}

Each element $H_{ij}[k]$ of the channel matrix represents the complex frequency response between the $j$-th transmit antenna and the $i$-th receive antenna on the $k$-th subcarrier. The matrix $\mathbf{H}[k]$ therefore captures both the multipath propagation effects and the spatial characteristics of the transmission environment.

In spatial multiplexing systems, the rank of $\mathbf{H}[k]$ determines the number of independent data streams that can be reliably transmitted on the corresponding subcarrier. Full-rank channel realizations allow the system to fully exploit spatial degrees of freedom, whereas rank deficiencies directly limit multiplexing capability. As a result, subcarrier-dependent variations of the channel matrix play a central role in the performance behavior of MIMO-OFDM systems \cite{tse2005fundamentals}.

\subsubsection{Detection Problem in MIMO-OFDM Systems}

The task of the receiver in spatial multiplexing mode is to recover the transmit vector $\mathbf{x}[k]$ from the received signal $\mathbf{y}[k]$. Due to the coupling of multiple data streams through the channel matrix, this constitutes a multidimensional detection problem.

Linear detection schemes such as zero-forcing (ZF) and minimum mean square error (MMSE) detection offer low computational complexity, while successive interference cancellation approaches, including the V-BLAST architecture, improve detection performance at the cost of increased complexity. In MIMO-OFDM systems, detection is typically performed independently on each subcarrier, preserving the modular structure introduced by OFDM.

\subsection{Channel Estimation and Linear Equalization in the Implemented MIMO-OFDM Receiver}
\label{sec:chest_mmse_impl}

Based on the per-subcarrier MIMO-OFDM signal model introduced in the previous section,
\begin{equation}
\mathbf{y}[k] = \mathbf{H}[k]\mathbf{x}[k] + \mathbf{n}[k],
\end{equation}
reliable recovery of the transmitted symbol vector $\mathbf{x}[k]$ requires knowledge of the frequency-domain MIMO channel matrix $\mathbf{H}[k]$ and an appropriate receiver-side equalization strategy. In the implemented audio-based demonstrator, channel estimation and equalization are therefore central components of the receiver chain, directly determining the quality of all subsequent symbol-domain and bit-level results reported in Chapter~4.

Rather than implementing a broad set of estimation and detection techniques, the demonstrator deliberately focuses on a compact and transparent processing chain that is well suited for analysis and teaching. Consequently, the receiver employs (i) a \emph{training-based least-squares / zero-forcing (LS/ZF)} channel estimator derived from a deterministic preamble, and (ii) a \emph{linear MMSE} equalizer applied independently on each OFDM subcarrier. Both choices are consistent with standard MIMO-OFDM theory and allow a clear connection between mathematical models, implementation, and observable performance metrics.

\subsubsection{Training-based channel estimation using a deterministic preamble}

Channel estimation in the demonstrator is based on a known preamble transmitted prior to the payload data. A deterministic sequence with constant amplitude and favorable correlation properties is used, enabling both robust synchronization and reliable channel estimation. After FFT processing, the preamble provides a known frequency-domain reference on each subcarrier.

Following time synchronization and cyclic-prefix removal, the received training symbols on subcarrier $k$ can be expressed as
\begin{equation}
\mathbf{Y}[k] = \mathbf{H}[k]\mathbf{X}_{\mathrm{p}}[k] + \mathbf{N}[k],
\end{equation}
where $\mathbf{X}_{\mathrm{p}}[k]$ denotes the known training symbol(s) transmitted during the preamble phase. In the implemented framing structure, transmit channels are sounded sequentially, such that on a given training OFDM symbol only one transmit channel is active. For the link between transmit channel $t$ and receive channel $r$, this reduces to
\begin{equation}
Y_{r,t}[k] = H_{r,t}[k]\,X_{\mathrm{p}}[k] + N_{r,t}[k].
\end{equation}

A least-squares estimate of the channel coefficient is obtained by direct inversion of the known training symbol,
\begin{equation}
\hat{H}_{r,t}[k] = \frac{Y_{r,t}[k]}{X_{\mathrm{p}}[k]}.
\end{equation}
This estimator corresponds to a zero-forcing inversion of the training symbol on each subcarrier and is therefore referred to as ``ZF'' in the demonstrator configuration. While more sophisticated estimators exist, the LS/ZF approach offers a particularly transparent mapping between the received signal and the estimated channel response, making it well suited for demonstrator-based analysis \cite{proakis2008digital,tse2005fundamentals}.

The resulting frequency-domain channel estimates are stored as channel transfer functions (CTF). A corresponding channel impulse response (CIR) is obtained by applying an inverse FFT across subcarriers for each loudspeaker--microphone pair, providing an intuitive time-domain view of multipath propagation effects.

\subsubsection{Linear MMSE equalization on each subcarrier}

Once an estimate $\hat{\mathbf{H}}[k]$ of the channel matrix is available, the receiver proceeds with spatial separation of the simultaneously transmitted data streams. In spatial multiplexing mode, this constitutes a linear detection problem on each subcarrier.

The demonstrator employs a linear minimum mean square error (MMSE) equalizer, which computes the estimate
\begin{equation}
\hat{\mathbf{x}}[k] = \mathbf{W}_{\mathrm{MMSE}}[k]\mathbf{y}[k],
\end{equation}
with the equalization matrix
\begin{equation}
\mathbf{W}_{\mathrm{MMSE}}[k]
=
\left(
\hat{\mathbf{H}}^{\mathrm{H}}[k]\hat{\mathbf{H}}[k] + \alpha\,\mathbf{I}
\right)^{-1}
\hat{\mathbf{H}}^{\mathrm{H}}[k],
\end{equation}
where $\alpha>0$ is a regularization parameter related to the inverse signal-to-noise ratio. Compared to zero-forcing detection, MMSE equalization mitigates excessive noise enhancement on subcarriers where the channel matrix is ill-conditioned, a situation that frequently arises in practical acoustic MIMO channels due to spatial correlation and frequency-selective fading \cite{tse2005fundamentals}.

The MMSE equalizer is applied independently on each subcarrier and OFDM block, producing a set of equalized complex symbols for all spatial streams. These symbols form the basis for constellation visualization, error vector magnitude (EVM) computation, and subsequent bit demapping as reported in Chapter~4.

\subsubsection{Relation to observable receiver metrics}

The combination of LS/ZF channel estimation and linear MMSE equalization establishes a direct link between the theoretical MIMO-OFDM model and the measurable quantities exposed by the demonstrator. Channel impulse and transfer responses reflect the estimated $\hat{\mathbf{H}}[k]$, channel rank and conditioning metrics are derived from its singular values, and constellation diagrams as well as EVM results directly visualize the quality of the MMSE-separated symbol streams. In this way, the chosen receiver structure enables a coherent and interpretable mapping from propagation effects to symbol-domain performance.